\chapter{Conclusion}

To summarize, it was successfully proven that concepts of the
artificial intelligence domain, namely convolutional neural networks,
can be applied in the context of fault detection within the CLAS12
drift chamber. As a result, a fully functional fault detection system
consisting of multiple CNNs was implemented and tested on real fault
data, only struggling with faults that show little contrast in terms
of wire activations. This issue can be remedied in the future however,
by presenting the classifier with more of these examples
during training, using real data containing blurred faults as
well. After all, a detection system based on deep learning
can only be as good as the data it was trained on.

Nevertheless, there are some further aspects that will have to be
covered in the future, before the fault detector can be used in
practice. In addition to detecting a fault in a superlayer, it is also
important to know exactly where the fault is located and which
particular wires are affected by the fault in order to be able to
precisely filter out the effects of faulty components on particle
measurements. These demands lead us from fault detection to the
domain of fault localization, which requires even more sophisticated
methods. One possible approach towards solving this problem could be
the YOLOv3 object detection system, a convolutional neural network
which is able to predict the position of an object within an image
\cite{yolo}. Further work could include applying such a concept to the
problem of fault localization, using the system implemented in
this thesis as a pre-stage classifier to determine which faults are
present in a superlayer before applying the more computationally
expensive fault localization procedure.

Lastly, it remains to be tested how the implemented fault detector
will perform when presented with large scale datasets that will accrue over
the next 20 years of approved CLAS12 operations. It is always possible
that some new surprises are present within the hundreds of petabytes
of experimental data that reside on the clusters of the CLAS12
project. Based on the promising results that originated from testing
and validating, performed
during this thesis, a high degree of confidence can be
justified when it comes to putting the fault detector to the test on
real time CLAS12 drift chamber data.
