\chapter*{Abstract}
\thispagestyle{empty}

When conducting physical experiments at particle detectors like it is
done within the CLAS12 project, highly precise methods of measuring
particle interactions are essential for success. It is therefore
mandatory to spot faulty components of a detector in real time to
account for their distortive effects on measurement accuracy. An
autonomous fault detection system for the CLAS12 drift chamber, the
most crucial part of the CLAS12 particle detector, is
designed in the scope of this thesis, using algorithms from the
domain of artificial intelligence, namely convolutional neural
networks. After laying out the necessary foundations of deep
learning, the fault detection system is implemented using the
deeplearning4j library and trained on simulated fault data. The
applicability of the system to the context of real experimental data
is justified by evaluating the performance of the fault detector when
presented with real world examples.
