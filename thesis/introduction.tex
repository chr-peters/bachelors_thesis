\chapter{Introduction}

In order for physical experiments like the ones conducted at the
CLAS12 particle detector to succeed, highly precise ways of measuring
particle interactions are required. It is therefore crucial to know in
time when parts of the experimental setup show signs of deficiency,
to reduce the effect of distortion introduced by these faulty
components.

At the CLAS12 particle detector, most of the measurements are
performed within the CLAS12 drift chamber, a subsystem of the detector
consisting of 24,192 wires, each designed to capture the presence of
particles. Due to the extreme conditions within the drift chamber, it
is common that single wires or collections thereof stop working
properly during an experimental run. In order to still be able to
perform accurate computations on the basis of these measurements,
faulty components have to be detected during runtime and their effects
have to be filtered out. This task requires huge amounts of data to be
processed in real time, which makes it infeasible to solve this
problem based on real time human interaction. An \textit{autonomous}
way of detecting faults is therefore required.

Luckily though, there is one field which has gained much popularity
in recent years that happens to lend it self particularly well to
the task of fault detection: \textit{Artificial
  Intelligence}. Many powerful algorithms have originated from this
domain, dominating the benchmarks of challenges such
as large scale image classification, even reaching human like
performance \cite{Russakovsky}.

The goal of this thesis is to apply algorithms of artificial
intelligence and machine learning, namely deep convolutional neural
networks, to the problem of autonomous fault detection within the
context of the CLAS12 drift chamber. 
