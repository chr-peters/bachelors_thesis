\chapter{Introduction}

In order for physical experiments at particle accelerators and
detectors to succeed, highly precise ways
of measuring particle interactions are required. It is therefore
crucial to recognize immediately when parts of the experimental setup
show signs of deficiency, because these faulty components can lead to
severe distortions in measurement accuracy.

At the CLAS12 particle detector, most of the measurements are
performed within the CLAS12 drift chamber, a subsystem of the detector
consisting of 24,192 wires, each designed to capture the presence of
particles to reconstruct a particle track. Due to the extreme
conditions within the drift chamber, it
is common that single wires or collections thereof stop working
properly during an experimental run. In order to still be able to
perform accurate computations on the basis of these measurements,
faulty components have to be detected during runtime and their effects
have to be filtered out. This task requires huge amounts of data to be
processed in real time, which makes it infeasible to solve this
problem based on real time human interaction. An \textit{autonomous}
approach of fault detection is therefore required.

Luckily though, there is one field which has gained much popularity
in recent years that happens to lend itself particularly well to
the task of fault detection: \textit{Artificial
  Intelligence}. Many powerful algorithms have originated from this
domain, dominating the benchmarks of challenges such
as large scale image classification, even reaching human like
performance \cite{Russakovsky}.

The goal of this thesis is to apply algorithms of artificial
intelligence and machine learning, namely deep convolutional neural
networks, to the problem of autonomous fault detection within the
context of the CLAS12 drift chamber.

After describing the environment presented by the CLAS12 detector and
the CLAS12 drift chamber in more detail, the fundamentals of deep
learning will be outlined and the most essential algorithms will be
derived. Upon this foundation, the central method utilized in the
scope of this thesis, convolutional neural networks, will be analyzed
to justify its applicability in the context of fault detection. The
\textit{deeplearning4j (DL4J)} framework will be used to implement the
fault detection algorithm which is evaluated afterwards on real fault
data.
