\chapter{Deep Learning Fundamentals}

\section{Artificial Neural Networks}

Artificial neural networks (ANNs) are a class of machine learning algorithms
that are loosely inspired by the structure of biological nervous
systems.
To be precise, each ANN consists of a collection of artificial neurons
that are connected with each other. The neurons are able to exchange
information along their connections.
A common way to arrange artificial neurons within a network is to organize
them in layers as depicted in \fref{fig:basic-network}.
\begin{figure}[h]
  \centering
  \resizebox{0.75\textwidth}{!}{\begin{neuralnetwork}[height=5]
  \tikzstyle{input neuron}=[neuron, draw, fill=white]
  \tikzstyle{hidden neuron}=[neuron, draw, fill=white]
  \tikzstyle{output neuron}=[neuron, draw, fill=white]
  \tikzstyle{link} = [->, shorten <=0pt, node distance=\nn@layerspacing, thin, draw=black];
  \inputlayer[count=4, bias=false, title=Input\\layer]
  \hiddenlayer[count=5, bias=false, title=Hidden\\layer] \linklayers
  \outputlayer[count=3, title=Output\\layer] \linklayers
\end{neuralnetwork}
}
  \caption{The structure of a simple ANN. The nodes represent the
    neurons, the edges represent their connections, also indicating
    the flow of information.}
  \label{fig:basic-network}
\end{figure}

When an artificial neuron receives a signal on some of its
incoming connections, it may elect to become active.\footnote{The
  details of this
  process are further illustrated in \fref{sec:artificial-neurons}.}
In this state it also influences all neurons it has an outgoing
connection to by passing a signal along their
channel. Those other neurons in turn may also elect to become
activated - this way a signal can propagate through the network along
the connecting edges.

Usually, each ANN consists of at least one layer of neurons that is
responsible for receiving signals from the environment - we call this
an \textit{input layer} (see \fref{fig:basic-network}). When these neurons
receive a signal from the environment, they propagate it to their
connected neighbors in the next layer. This process repeats until the
\textit{output layer} is reached. The neurons in this layer represent the
output of the whole network. Each layer in between is called a \textit{hidden
layer} because there is no direct communication between the neurons in
this layer and the environment.

\subsection{Modeling Artificial Neurons}
\label{sec:artificial-neurons}

\section{The Multilayer Perceptron}

\section{Deep Networks}
